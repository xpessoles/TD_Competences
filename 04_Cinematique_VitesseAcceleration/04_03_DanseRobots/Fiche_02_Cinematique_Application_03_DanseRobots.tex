%%%% Paramétrage du TD %%%%
\def\xxactivite{Révisions \ifprof -- Corrigé \else \fi} % \normalsize \vspace{-.4cm}
\def\xxauteur{\textsl{Xavier Pessoles}}


\def\xxnumchapitre{Révision cinématique \vspace{.2cm}}
\def\xxchapitre{\hspace{.12cm} Résolution cinématique}
\def\xxonglet{\textsf{Rév -- Stat}}
\def\xxactivite{TD 03}
\def\xxauteur{\textsl{Xavier Pessoles}}

\def\xxpied{%
Révision cinématique \\
Fiche  2 -- \xxactivite%
}


\def\xxtitreexo{Danse avec les robots}
\def\xxsourceexo{\hspace{.2cm} \footnotesize{ICNA 2017}}


\def\xxcompetences{%
\textsl{%
\textbf{Savoirs et compétences :}\\} \vspace{-.5cm}
%\begin{itemize}
%\item \textit{Res2.C18} : principe fondamental de la statique;
%\item \textit{Res2.C19} : équilibre d’un solide, d’un ensemble de solides;
%\item \textit{Res2.C20} : théorème des actions réciproques.
%\end{itemize}
}


\def\xxfigures{
\includegraphics[width=.45\textwidth]{fig_01}
}%figues de la page de garde




\iflivret
\input{../../style/new_pagegarde}
\else
\input{../../style/new_pagegarde}
\fi
\setlength{\columnseprule}{.1pt}

\pagestyle{fancy}
\thispagestyle{plain}

\ifprof
\vspace{5cm}
\else
\vspace{4cm}
\fi

\def\columnseprulecolor{\color{ocre}}
\setlength{\columnseprule}{0.4pt} 

%%%%%%%%%%%%%%%%%%%%%%%

\setcounter{exo}{0}



\ifprof
\else
\begin{multicols}{2}
\fi


<< Danse avec les robots >> est une attraction du Futuroscope de Poitiers. Le principe consiste à attacher deux personnes au bout d'un bras de robot 5 axes. Les personnes sont ainsi remués au rythme de la musique.

On appelle nacelle l'ensemble de solides composé des sièges, des harnais de sécurité et des 2 volontaires. 

\begin{center}
\includegraphics[width=\linewidth]{fig_02}
\end{center}


On donne sur la figure suivant le schéma cinématique spatial d'un des robots avec le paramétrage associé aux différents solides et aux liaisons. 

\begin{center}
\includegraphics[width=\linewidth]{fig_03}
\end{center}

L'ensemble des repères sont considérés orthonormés directs.


\begin{itemize}
\item On note $\rep{0}=\repere{O_0}{x_0}{y_0}{z_0}$ le repère supposé galiléen associé au sol de la salle de spectacle, appelé bâti \textbf{0}.
\item On note $\rep{1}=\repere{O_0}{x_1}{y_1}{z_1}$ le repère associé à la chaise \textbf{1} et $\theta_1=\angl{x_0}{x_1}=\angl{y_0}{y_1}$ l'angle de rotation de la chaise \textbf{1} par rapport au bâti \textbf{0}.
\item On note $\rep{2}=\repere{A}{x_2}{y_2}{z_2}$ le repère associé à l'épaule \textbf{2}, $\vect{O_0A} = a\vect{z_0} + b\vect{x_1}$ et $\theta_2=\angl{x_1}{x_2}=\angl{z_1}{z_2}$ l'angle de rotation de l'épaule \textbf{2} par rapport à la chaise \textbf{1}.
\item On note $\rep{3}=\repere{B}{x_3}{y_3}{z_3}$ le repère associé à l'avant-bras \textbf{3}, $\vect{AB} = c\vect{x_2}$ et $\theta_3=\angl{x_2}{x_3}=\angl{z_2}{z_3}$ l'angle de rotation de l'avant-bras \textbf{3} par rapport à l'épaule \textbf{2}.
\item On note $\rep{4}=\repere{C}{x_4}{y_4}{z_4}$ le repère associé au bras \textbf{4}, $\vect{BC} = d\vect{x_3}$ et $\theta_4=\angl{x_3}{x_4}=\angl{y_3}{y_4}$ l'angle de rotation du bras \textbf{4} par rapport à l'avant-bras \textbf{3}.
\item On note $\rep{5}=\repere{D}{x_5}{y_5}{z_5}$ le repère associé à la nacelle \textbf{5}, $\vect{CD} = e\vect{x_4}$ et $\theta_5=\angl{y_4}{y_5}=\angl{z_4}{z_5}$ l'angle de rotation de la nacelle \textbf{5} par rapport au bras \textbf{4}.
\end{itemize}
Le centre de gravité de la nacelle \textbf{5} (siège + volontaire + harnais) est tel que $\vect{DG}=f\vect{x_4}+h\vect{z_5}$. 

On définit la position du point $G$ dans la base $\mathcal{B}_0 = \base{x_0}{y_0}{z_0}$ telle que $\vect{O_0 G} = x\vect{x_0}+y\vect{y_0}+z\vect{z_0}$.
 
\subparagraph{}
\textit{Tracer les figures planes de changement de repère.}
\ifprof
\begin{corrige}
\end{corrige}
\else\fi


\subparagraph{}
\textit{Exprimer la position du point $G$ suivant $\vect{x_0}$.}
\ifprof
\begin{corrige}
\end{corrige}
\else\fi

\begin{obj}
Valider que l'exigence d'accélération est satisfaite : l'accélération ressentie doit être au maximum de \SI{3,5}{g}.
\end{obj}


%L'accélération ressentie par les volontaires, notée $\vect{\Gamma}_G}=\vecta{G}{5}{0}$ avec $\vect{g}=-g\vect{z_0}$.
%On limite l'étude dans un premier temps au cas où $\theta_1 = \text{cte}$ et donc $\dot{\theta}_1=0$. 

\subparagraph{}
\textit{Exprimer la vitesse du point $G$ dans son mouvement par rapport au repère galiléen associé à \textbf{0}, notée $\vectv{G}{5}{0}$.}
\ifprof
\begin{corrige}
\end{corrige}
\else\fi

On limite désormais l'étude dans au cas où $\dot{\theta}_2 = \SI{1,45}{rad.s^{-1}}$, ${\theta}_3={\theta}_4={\theta}_5=0$.


\subparagraph{}
\textit{Exprimer l’accélération du point $G$ dans son mouvement par rapport au repère galiléen associé à 0, notée $\vectg{G}{5}{0}$.}
\ifprof
\begin{corrige}
\end{corrige}
\else\fi

\subparagraph{}
\textit{Conclure quant au respect de l'exigence d'accélération ressentie. }
\ifprof
\begin{corrige}
\end{corrige}
\else\fi



\ifprof
\else
\end{multicols}
\fi


