\section{Validation des capacités de positionnement du système de déploiement}

\begin{obj}
Vérifier l’exigence 002 « Position de la pince » afin que le point de préhension $P$ du 
système de déploiement DPL puisse être défini à partir de deux coordonnées articulaires.
\end{obj}
%
%\question{Établir la relation vectorielle entre $X_P$, $Y_P$, $L$ et $\vx{0}$, $\vy{0}$, $\vx{1}$ et $\vx{2}$. }
%
%\question{Projeter la relation précédente selon $\vx{0}$ et $\vy{0}$, puis donner les deux équations scalaires 
%correspondantes.}

\question{Exprimer $\theta_1$ et $\theta_2$ en fonction de $X_P$, $Y_P$ et $L$. Conclure quant au respect de «l’exigence 
002».}


\section{Validation du non-dépassement de la vitesse de la sphère SEIS}

\begin{obj}
Valider l’exigence 003 << Vitesse de la pince >> quand la sphère SEIS se déplace en 
translation afin de conserver toujours la même orientation : la vitesse de la pince ne doit pas excéder $\SI{20}{mm/s}$.
\end{obj}

On note $\vectv{M}{S}{R}$ est le vecteur vitesse du point $M$ appartenant au solide $S$ par rapport à $R$.s

%\question{Déterminer l’expression de la vitesse du point $P$, appartenant à l’avant-bras 2, par rapport à $\rep{0}$  en fonction de $\theta_1$, $\theta_2$ et $L$.}

\question{Déterminer la valeur maximale du taux de rotation $|| \vecto{1}{0} ||$  pour que l’avant-bras 2 suive un 
mouvement de translation circulaire par rapport à $\rep{0}$ en respectant l’exigence 003 << Vitesse de la pince >>.}

