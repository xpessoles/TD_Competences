\section{Validation des capacités dynamiques du système de déploiement}

\begin{obj}
Déterminer le couple du moto-réducteur $\indice{M}{01}$ qui permet la manipulation de la sphère SEIS par le système de déploiement.
\end{obj}

La figure \ref{fig_07} présente la schématisation du bras de déploiement, noté $\Sigma = \{1,2,S\}$.

%%Q08 
%\question{Justifier que la matrice d'inertie du bras 1, en son centre d'inertie $G_1$, est de la forme $\inertie{G_1}{O_1}=\matinertie{I_1}{J_1}{K_1}{0}{0}{0}{\rep{1}}$.}

\question{Exprimer le moment d'inertie $\indice{K}{O1}$ du bras 1 au point $O$ suivant $\vect{z_0}$ en fonction des paramètres cinétiques.}

\question{Exprimer le moment d'inertie $K_{O\Sigma}$ de l'ensemble $\Sigma$ au point $O$ suivant $\vect{z_0}$ en fonction des paramètres cinétiques.}

On considère, pour la suite, que le moteur $\indice{M}{02}$ est à l’arrêt dans la position $\theta_2 = 0$ et que seul le moteur $\indice{M}{01}$ est en fonctionnement.

\question{ Pour effectuer une modélisation dynamique du système, établir l’équation donnant le couple, 
noté $\indice{C}{01}$, du moteur $\indice{M}{01}$ en fonction des paramètres cinétiques du système de déploiement.  Préciser clairement le système isolé ainsi que le principe/théorème utilisé.}


Des calculs amènent à considérer que la valeur de $K_{O\Sigma}$ est très faible et donc pratiquement 
négligeable.

\question{Donner l’expression de l’équation précédente limitée au voisinage de la position du système de 
déploiement la plus défavorable.}
