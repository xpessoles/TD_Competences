\section{Étude de l’asservissement en position du vérin}

\begin{obj}
Choisir un correcteur approprié permettant de satisfaire le cahier des charges vis-à-vis des 
exigences concernant l’asservissement en position du vérin électrique suivant l’axe $\vect{y_0}$
conformément à la figure 11.
\end{obj}

\begin{figure}[!h]
\centering
\includegraphics[width=.8\linewidth]{fig_11.png}
\caption{Déplacement de la tige du vérin\label{fig_11}}
\end{figure}

La mesure de la distance est obtenue grâce à un capteur à ultrason permettant de délivrer, sous la 
forme d’impulsions, une image de la distance entre la structure sur SEIS et le sol. Cette information 
est ensuite traitée afin de générer un signal image de la distance parcourue par la tige du vérin.

%L’étude précédente a permis d’obtenir un modèle de comportement de la MCC intégré dans le schéma bloc de l’asservissement présenté en figure \ref{fig_12} pour lequel $C_r(p)=0$
%
%\begin{figure}[!h]
%\centering
%\includegraphics[width=\linewidth]{fig_12.png}
%\caption{Schéma-bloc de l’asservissement en position du vérin électrique\label{fig_12}}
%\end{figure}

\textbf{Notations et spécifications :}
\begin{multicols}{2}
\begin{itemize}
\item gain du capteur: $\indice{K}{capt} = \SI{588}{impulsions/m}$;
\item gain de l’ensemble réducteur et vis-écrou : $\indice{K}{red} = \SI{19.1e-6}{m/rad}$;
\item vitesse linéaire de la tige du vérin : $V(t)$ en \si{m.s^{-1}};
\item déplacement linéaire de la tige du vérin : $d(t)$ en $\si{V}$;
\item correcteur: $C_0(p)$;
\item gain du hacheur: $\indice{K}{H} = 1,163$.
\end{itemize}
\end{multicols}

Pour toute la suite du sujet, on considère : $\indice{C}{r}(p)=0$. Tout d'abord, le correcteur est consdéré unitaire $\indice{C}{0}(p)=1$.

%\question{Donner l'expression littérale de la fonction de transfert en b}
\question{Proposer un schéma-bloc permettant de modéliser l'asservissement du déplacement de la tige.}

%\question{Donner l’expression littérale de $M(p)$ et, pour garantir un bon asservissement, l’expression 
%littéralede $\indice{K}{adapt}$.}

\question{ Déterminer l’expression littérale de la fonction de transfert en boucle ouverte $\indice{G}{BO}(p)$ et mettre celle-ci sous forme canonique. Donner la classe de cette fonction de transfert. En déduire
 la précision du système.}
 
  On donne l’expression numérique dela fonction de transfert en boucle ouverte :
  $\indice{G}{BO}(p)=\dfrac{0,0112}{p\left( 0,00028 p +1\right)}$.

\question{Tracer les diagrammes de Bode asymptotiques et réels de la fonction de transfert $\indice{G}{BO}(p)$ sur le DR3. En déduire la marge de phase de l’asservissement en effectuant toutes les constructions 
graphiques nécessaires. Conclure sur le respect de l’exigence 006 «Stabilité ».}

\begin{figure}[!h]
\centering
\includegraphics[width=.8\linewidth]{dr_03.png}
\caption{Diagramme de Bode \label{dr_03}}
\end{figure}

 On désire quantifier la rapidité du système à la suite d’une sollicitation en échelon. On donne les 
relations permettant de calculer le temps de réponse à 5\%, noté $\indice{t}{r5\%}$, pour un système d’ordre deux (avec $\xi$ le facteur d’amortissement et $\omega_0$ la pulsation propredu système non amorti):

$$
\left\{ 
\begin{array}{l}
\xi < \dfrac{1}{\sqrt{2}}, \indice{t}{r5\%} \simeq \dfrac{3\xi}{ \omega_0} \\
\xi > \dfrac{1}{\sqrt{2}}, \indice{t}{r5\%} \simeq \dfrac{6\xi }{\omega_0} \\
\end{array}
\right.
$$ 


\question{ Déterminer et calculer les paramètres caractéristiques de la fonction de transfert en boucle 
fermée $\indice{G}{BF}=\dfrac{D(p)}{D_c(p)}$. En déduire le temps de réponse de l’asservissement en vitesse. Conclure sur le respect de l’exigence 004 «Rapidité».}

Afin d’améliorer les performances de l’asservissement, on choisit un correcteur proportionnel de gain 
$\indice{K}{D}$ tel que $C_0(p) = \indice{K}{D}$. La valeur numérique du gain sera déterminéeà partir de deux méthodes :
\begin{itemize}
\item approche graphique, à partir de la marge de phase (maîtrise de la stabilité);
\item approche analytique, à partir d’un comportement imposé.
\end{itemize}

\question{À partir de constructions graphiques sur la figure \ref{dr_03}, donner la valeur du gain du correcteur $\indice{K}{D1}$, permettant de garantir une marge de phase supérieure à 70\degres. La valeur de $\indice{K}{D1}$ vous paraît-elle pertinente et réaliste ?}

 On impose un temps de réponse à 5\% de \SI{5}{s} et un facteur d’amortissement $\xi$ supérieur à 1.
 On donne l’expression numérique de $\indice{G}{BF}(p)$ avec un correcteur de gain $\indice{K}{D}$ :
 $\indice{G}{BF}(p)=\dfrac{1}{\dfrac{0,025}{K_D}p^2+\dfrac{89}{K_D}p +1}$.
 
 
\question{À partir des équations liant le temps de réponse, le facteur d’amortissement et la pulsation 
propre ainsi que de l’expression numérique de $\indice{G}{BF}(p)$, donner une expression liant $t_{r5\%}$  et $\indice{K}{D2}$. En déduire la valeur de $\indice{K}{D2}$ permettant de respecter la contrainte imposée en termes de rapidité.}


 On donne ci-dessous les tracés de la sortie du système asservi à la suite d’un échelon de consigne de 
\SI{10}{cm} pour $\indice{K}{D1} = \num{220000}$ et $\indice{K}{D2} = 53$.


\begin{figure}[!h]
\centering
\includegraphics[width=\linewidth]{fig_13.png}
\caption{Réponses indicielles du système asservi \label{fig_13}}
\end{figure}

\question{Commenter les courbes (respect des exigences) et choisir le correcteur qui vous paraît le plus 
pertinent.}
